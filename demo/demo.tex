\documentclass{article}
\usepackage{commands}
\begin{document}

% demo 1: def of continuity
Let \( U \) be an open subset of \( \mathbb{R} \).
A function \( f:U \to \mathbb{R} \) is said to be \textit{continuous at } \( x_{0} \) if for any \( \epsilon > 0 \) there is \( \delta > 0 \) such that

\begin{equation*}
	\abs{x - x_{0}} \implies \abs{f(x) - f(x_{0})} < \epsilon.
\end{equation*}

% demo 2: dual basis
Let \( \left( e_{1}, \ldots, e_{n} \right) \) be a basis of a vector space \( V \) and let \( \left( f^{1}, \ldots, f^{n} \right) \) is the corresponding dual basis.

% demo 3: taylor expansion

\begin{equation*}
	\exp x = \sum_{n = 1}^{\infty} \frac{x^{n}}{n!} \\
	= \lim_{n \to \infty} \left( 1 + \frac{x}{n}  \right)^{n}
\end{equation*}

%demo 4: tensor isomorphism
\( V \otimes W \simeq W \otimes V \).
Indeed, the map \( V \otimes W \ni x \otimes y \mapsto y \otimes x \in W \otimes V \) is an isomorphism.

\end{document}